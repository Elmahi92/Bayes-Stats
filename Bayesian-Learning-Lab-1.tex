% Options for packages loaded elsewhere
\PassOptionsToPackage{unicode}{hyperref}
\PassOptionsToPackage{hyphens}{url}
%
\documentclass[
]{article}
\usepackage{amsmath,amssymb}
\usepackage{lmodern}
\usepackage{iftex}
\ifPDFTeX
  \usepackage[T1]{fontenc}
  \usepackage[utf8]{inputenc}
  \usepackage{textcomp} % provide euro and other symbols
\else % if luatex or xetex
  \usepackage{unicode-math}
  \defaultfontfeatures{Scale=MatchLowercase}
  \defaultfontfeatures[\rmfamily]{Ligatures=TeX,Scale=1}
\fi
% Use upquote if available, for straight quotes in verbatim environments
\IfFileExists{upquote.sty}{\usepackage{upquote}}{}
\IfFileExists{microtype.sty}{% use microtype if available
  \usepackage[]{microtype}
  \UseMicrotypeSet[protrusion]{basicmath} % disable protrusion for tt fonts
}{}
\makeatletter
\@ifundefined{KOMAClassName}{% if non-KOMA class
  \IfFileExists{parskip.sty}{%
    \usepackage{parskip}
  }{% else
    \setlength{\parindent}{0pt}
    \setlength{\parskip}{6pt plus 2pt minus 1pt}}
}{% if KOMA class
  \KOMAoptions{parskip=half}}
\makeatother
\usepackage{xcolor}
\usepackage[margin=1in]{geometry}
\usepackage{color}
\usepackage{fancyvrb}
\newcommand{\VerbBar}{|}
\newcommand{\VERB}{\Verb[commandchars=\\\{\}]}
\DefineVerbatimEnvironment{Highlighting}{Verbatim}{commandchars=\\\{\}}
% Add ',fontsize=\small' for more characters per line
\usepackage{framed}
\definecolor{shadecolor}{RGB}{248,248,248}
\newenvironment{Shaded}{\begin{snugshade}}{\end{snugshade}}
\newcommand{\AlertTok}[1]{\textcolor[rgb]{0.94,0.16,0.16}{#1}}
\newcommand{\AnnotationTok}[1]{\textcolor[rgb]{0.56,0.35,0.01}{\textbf{\textit{#1}}}}
\newcommand{\AttributeTok}[1]{\textcolor[rgb]{0.77,0.63,0.00}{#1}}
\newcommand{\BaseNTok}[1]{\textcolor[rgb]{0.00,0.00,0.81}{#1}}
\newcommand{\BuiltInTok}[1]{#1}
\newcommand{\CharTok}[1]{\textcolor[rgb]{0.31,0.60,0.02}{#1}}
\newcommand{\CommentTok}[1]{\textcolor[rgb]{0.56,0.35,0.01}{\textit{#1}}}
\newcommand{\CommentVarTok}[1]{\textcolor[rgb]{0.56,0.35,0.01}{\textbf{\textit{#1}}}}
\newcommand{\ConstantTok}[1]{\textcolor[rgb]{0.00,0.00,0.00}{#1}}
\newcommand{\ControlFlowTok}[1]{\textcolor[rgb]{0.13,0.29,0.53}{\textbf{#1}}}
\newcommand{\DataTypeTok}[1]{\textcolor[rgb]{0.13,0.29,0.53}{#1}}
\newcommand{\DecValTok}[1]{\textcolor[rgb]{0.00,0.00,0.81}{#1}}
\newcommand{\DocumentationTok}[1]{\textcolor[rgb]{0.56,0.35,0.01}{\textbf{\textit{#1}}}}
\newcommand{\ErrorTok}[1]{\textcolor[rgb]{0.64,0.00,0.00}{\textbf{#1}}}
\newcommand{\ExtensionTok}[1]{#1}
\newcommand{\FloatTok}[1]{\textcolor[rgb]{0.00,0.00,0.81}{#1}}
\newcommand{\FunctionTok}[1]{\textcolor[rgb]{0.00,0.00,0.00}{#1}}
\newcommand{\ImportTok}[1]{#1}
\newcommand{\InformationTok}[1]{\textcolor[rgb]{0.56,0.35,0.01}{\textbf{\textit{#1}}}}
\newcommand{\KeywordTok}[1]{\textcolor[rgb]{0.13,0.29,0.53}{\textbf{#1}}}
\newcommand{\NormalTok}[1]{#1}
\newcommand{\OperatorTok}[1]{\textcolor[rgb]{0.81,0.36,0.00}{\textbf{#1}}}
\newcommand{\OtherTok}[1]{\textcolor[rgb]{0.56,0.35,0.01}{#1}}
\newcommand{\PreprocessorTok}[1]{\textcolor[rgb]{0.56,0.35,0.01}{\textit{#1}}}
\newcommand{\RegionMarkerTok}[1]{#1}
\newcommand{\SpecialCharTok}[1]{\textcolor[rgb]{0.00,0.00,0.00}{#1}}
\newcommand{\SpecialStringTok}[1]{\textcolor[rgb]{0.31,0.60,0.02}{#1}}
\newcommand{\StringTok}[1]{\textcolor[rgb]{0.31,0.60,0.02}{#1}}
\newcommand{\VariableTok}[1]{\textcolor[rgb]{0.00,0.00,0.00}{#1}}
\newcommand{\VerbatimStringTok}[1]{\textcolor[rgb]{0.31,0.60,0.02}{#1}}
\newcommand{\WarningTok}[1]{\textcolor[rgb]{0.56,0.35,0.01}{\textbf{\textit{#1}}}}
\usepackage{graphicx}
\makeatletter
\def\maxwidth{\ifdim\Gin@nat@width>\linewidth\linewidth\else\Gin@nat@width\fi}
\def\maxheight{\ifdim\Gin@nat@height>\textheight\textheight\else\Gin@nat@height\fi}
\makeatother
% Scale images if necessary, so that they will not overflow the page
% margins by default, and it is still possible to overwrite the defaults
% using explicit options in \includegraphics[width, height, ...]{}
\setkeys{Gin}{width=\maxwidth,height=\maxheight,keepaspectratio}
% Set default figure placement to htbp
\makeatletter
\def\fps@figure{htbp}
\makeatother
\setlength{\emergencystretch}{3em} % prevent overfull lines
\providecommand{\tightlist}{%
  \setlength{\itemsep}{0pt}\setlength{\parskip}{0pt}}
\setcounter{secnumdepth}{-\maxdimen} % remove section numbering
\ifLuaTeX
  \usepackage{selnolig}  % disable illegal ligatures
\fi
\IfFileExists{bookmark.sty}{\usepackage{bookmark}}{\usepackage{hyperref}}
\IfFileExists{xurl.sty}{\usepackage{xurl}}{} % add URL line breaks if available
\urlstyle{same} % disable monospaced font for URLs
\hypersetup{
  pdftitle={Bayesian Learning Lab 1},
  pdfauthor={Mohamed Ali},
  hidelinks,
  pdfcreator={LaTeX via pandoc}}

\title{Bayesian Learning Lab 1}
\author{Mohamed Ali}
\date{2023-05-06}

\begin{document}
\maketitle

\hypertarget{question-1-daniel-bernoulli}{%
\subsection{Question 1 Daniel
Bernoulli}\label{question-1-daniel-bernoulli}}

First we calculate the mean and the standar deviation from the below
equation to compare the with sample means and standar deviations of
\(\theta\) as function of the accumulating number of drawn values. \[
E(\theta|y) \ =\ a/a+b
\] \[
E(\theta|y) \ =\ ab/(a+b)^2(a+b+1)
\] \# A \#\#\# Figure 1.1

The figure below shows how the values of the Mean and Sd converges to
the true values as the number of draws grows large

\begin{Shaded}
\begin{Highlighting}[]
\NormalTok{ber\_fun}\OtherTok{\textless{}{-}}\ControlFlowTok{function}\NormalTok{(n\_d,n,s,a,b) \{}
  \CommentTok{\#Initial Value of the function givan from the question}
\NormalTok{  t\_n }\OtherTok{=}\NormalTok{ n}
\NormalTok{  s   }\OtherTok{=}\NormalTok{ s}
\NormalTok{  f   }\OtherTok{=}\NormalTok{ n}\SpecialCharTok{{-}}\NormalTok{s}
\NormalTok{  a   }\OtherTok{=}\NormalTok{ a}
\NormalTok{  b   }\OtherTok{=}\NormalTok{ b}
  \CommentTok{\#Beta(alpha+s,beta+s)}
\NormalTok{  a\_new }\OtherTok{=}\NormalTok{ a}\SpecialCharTok{+}\NormalTok{s}
\NormalTok{  b\_new }\OtherTok{=}\NormalTok{ b}\SpecialCharTok{+}\NormalTok{f}
\NormalTok{  Mean\_true}\OtherTok{=}\NormalTok{ a\_new}\SpecialCharTok{/}\NormalTok{(a\_new}\SpecialCharTok{+}\NormalTok{b\_new)}
  \CommentTok{\#we take the sqrt to get the Sd insted of the Var}
\NormalTok{  Sd\_true}\OtherTok{=}  \FunctionTok{sqrt}\NormalTok{((a\_new}\SpecialCharTok{*}\NormalTok{b\_new)}\SpecialCharTok{/}\NormalTok{(((a\_new}\SpecialCharTok{+}\NormalTok{b\_new)}\SpecialCharTok{\^{}}\DecValTok{2}\NormalTok{) }\SpecialCharTok{*}\NormalTok{ (a\_new}\SpecialCharTok{+}\NormalTok{b\_new}\SpecialCharTok{+}\DecValTok{1}\NormalTok{)))}
  \CommentTok{\# Beta(alpha+s,beta+s)}
\NormalTok{  mean\_theta }\OtherTok{=} \FunctionTok{c}\NormalTok{()}
\NormalTok{  sd\_theta }\OtherTok{=} \FunctionTok{c}\NormalTok{()}
\NormalTok{  n\_draws }\OtherTok{=} \DecValTok{1}\SpecialCharTok{:}\NormalTok{n\_d}
  \CommentTok{\#For loop to fill the values of mean and sd based on the draws }
  \ControlFlowTok{for}\NormalTok{ (i }\ControlFlowTok{in} \DecValTok{1}\SpecialCharTok{:}\NormalTok{n\_d)\{}
\NormalTok{    mean\_theta[i]}\OtherTok{=}\FunctionTok{mean}\NormalTok{(}\FunctionTok{rbeta}\NormalTok{(i,a\_new,b\_new))}
\NormalTok{    sd\_theta[i]}\OtherTok{=}\FunctionTok{sd}\NormalTok{(}\FunctionTok{rbeta}\NormalTok{(i,a\_new,b\_new))\}}
  \CommentTok{\#Binding everything togther}
\NormalTok{  df}\OtherTok{\textless{}{-}}\FunctionTok{cbind.data.frame}\NormalTok{(n\_draws,mean\_theta,sd\_theta)}
  \CommentTok{\#Plot of the mean }
\NormalTok{  mean}\OtherTok{\textless{}{-}}\FunctionTok{ggplot}\NormalTok{(df,}\FunctionTok{aes}\NormalTok{(}\AttributeTok{x=}\NormalTok{n\_draws))}\SpecialCharTok{+}\FunctionTok{geom\_line}\NormalTok{(}\FunctionTok{aes}\NormalTok{(}\AttributeTok{y=}\NormalTok{mean\_theta)}
\NormalTok{                                            , }\AttributeTok{color=}\StringTok{\textquotesingle{}\#FCA311\textquotesingle{}}\NormalTok{, }\AttributeTok{size=}\NormalTok{.}\DecValTok{8}\NormalTok{)}\SpecialCharTok{+}
              \FunctionTok{geom\_line}\NormalTok{(}\FunctionTok{aes}\NormalTok{(}\AttributeTok{y=}\NormalTok{Mean\_true), }\AttributeTok{color=}\StringTok{\textquotesingle{}\#14213D\textquotesingle{}}\NormalTok{,}\AttributeTok{linetype=}\DecValTok{3}\NormalTok{)}\SpecialCharTok{+}
              \FunctionTok{annotate}\NormalTok{(}\AttributeTok{geom =} \StringTok{"text"}\NormalTok{, }\AttributeTok{x =} \DecValTok{8}\NormalTok{, }\AttributeTok{y =}\NormalTok{ Mean\_true, }
              \AttributeTok{label =} \FunctionTok{paste0}\NormalTok{(}\FunctionTok{format}\NormalTok{(}\FunctionTok{round}\NormalTok{(Mean\_true, }\DecValTok{3}\NormalTok{), }\AttributeTok{nsmall =} \DecValTok{3}\NormalTok{)))}\SpecialCharTok{+}
              \FunctionTok{labs}\NormalTok{(}\AttributeTok{title =} \StringTok{\textquotesingle{}Sample Means and SD of theta \textquotesingle{}}\NormalTok{,}
         \AttributeTok{subtitle =} \StringTok{\textquotesingle{}As a function of the accumulating number of drawn values\textquotesingle{}}\NormalTok{,}
                          \AttributeTok{x=} \StringTok{\textquotesingle{} \textquotesingle{}}\NormalTok{, }\AttributeTok{y=}\StringTok{\textquotesingle{}Sample Mean\textquotesingle{}}\NormalTok{)}\SpecialCharTok{+} \FunctionTok{theme\_classic}\NormalTok{()}
  \CommentTok{\#Plot of the Sd}
\NormalTok{  sd}\OtherTok{\textless{}{-}}\FunctionTok{ggplot}\NormalTok{(df,}\FunctionTok{aes}\NormalTok{(}\AttributeTok{x=}\NormalTok{n\_draws))}\SpecialCharTok{+}\FunctionTok{geom\_line}\NormalTok{(}\FunctionTok{aes}\NormalTok{(}\AttributeTok{y=}\NormalTok{sd\_theta),}
                                          \AttributeTok{color=}\StringTok{\textquotesingle{}\#FCA311\textquotesingle{}}\NormalTok{, }\AttributeTok{size=}\NormalTok{.}\DecValTok{8}\NormalTok{)}\SpecialCharTok{+}
              \FunctionTok{geom\_line}\NormalTok{(}\FunctionTok{aes}\NormalTok{(}\AttributeTok{y=}\NormalTok{Sd\_true), }\AttributeTok{color=}\StringTok{\textquotesingle{}\#14213D\textquotesingle{}}\NormalTok{,}\AttributeTok{linetype=}\DecValTok{3}\NormalTok{)}\SpecialCharTok{+}
              \FunctionTok{annotate}\NormalTok{(}\AttributeTok{geom =} \StringTok{"text"}\NormalTok{, }\AttributeTok{x =} \DecValTok{8}\NormalTok{, }\AttributeTok{y =}\NormalTok{ Sd\_true,}
              \AttributeTok{label =} \FunctionTok{paste0}\NormalTok{(}\FunctionTok{format}\NormalTok{(}\FunctionTok{round}\NormalTok{(Sd\_true, }\DecValTok{3}\NormalTok{), }\AttributeTok{nsmall =} \DecValTok{3}\NormalTok{)))}\SpecialCharTok{+}
              \FunctionTok{labs}\NormalTok{(}\AttributeTok{x=} \StringTok{\textquotesingle{}Number of draws\textquotesingle{}}\NormalTok{, }\AttributeTok{y=}\StringTok{\textquotesingle{}Standard Deviation\textquotesingle{}}\NormalTok{)}\SpecialCharTok{+}
              \FunctionTok{theme\_classic}\NormalTok{()}
  \CommentTok{\#grid.arrange(mean,sd)}
  \FunctionTok{return}\NormalTok{(}\FunctionTok{grid.arrange}\NormalTok{(mean,sd))}
\NormalTok{\}}


\FunctionTok{ber\_fun}\NormalTok{(}\DecValTok{100}\NormalTok{,}\DecValTok{70}\NormalTok{,}\DecValTok{22}\NormalTok{,}\DecValTok{8}\NormalTok{,}\DecValTok{8}\NormalTok{)}
\end{Highlighting}
\end{Shaded}

\begin{verbatim}
## Warning: Removed 1 row(s) containing missing values (geom_path).
\end{verbatim}

\includegraphics{Bayesian-Learning-Lab-1_files/figure-latex/ber_fun-1.pdf}

\hypertarget{b}{%
\section{B}\label{b}}

First we find the value from the beta posterior using the function
\(pbeta\) we get the value of theta which can be used to compute the
probability that a random variable from a beta distribution is less than
or equal to a given value, or greater than a given value, depending on
the value of the lower.tail argument. in our case we use lower.tail as
False because we need the values grater than.

\begin{verbatim}
## [[1]]
## [1] "random values from the posterior with the given condition 0.84"
## 
## [[2]]
## [1] "The exact value from the Beat posterior 0.828593587338586"
\end{verbatim}

\hypertarget{c}{%
\section{C}\label{c}}

Now we want to draw 10000 random values from the posterior of the odds

\includegraphics{Bayesian-Learning-Lab-1_files/figure-latex/grp-1.pdf}

\hypertarget{question-2-log-normal-distribution-and-gini-coefficient}{%
\subsection{Question 2 Log-Normal distribution and Gini
Coefficient}\label{question-2-log-normal-distribution-and-gini-coefficient}}

\hypertarget{a}{%
\section{A}\label{a}}

First we find the value of tau, the question assume that we have 8
randomly selected persons thus we have n = 8 and we have the value of mu
= 3.6 Given, to find the value of tau we use this equation:
\[\tau^2\ =\ \sum_1^n(log\ y_i \ - \mu)^2/n\]'. \emph{Note} A
non-informative prior is a prior distribution that is chosen to express
little to no prior information about the parameters. Non-informative
priors are often chosen to avoid introducing bias or strong assumptions
into the model, and to allow the data to have a greater influence on the
posterior distribution. Examples of non-informative priors include the
uniform distribution, the Jeffreys prior, and the reference prior. It's
important to note that a non-informative prior is not necessarily a
prior with no information at all, but rather one that expresses a
minimal amount of information that is consistent with our knowledge and
beliefs before observing the data. In practice, the choice of prior
distribution often depends on the specific problem and the available
prior knowledge.

in the question we have inverse chi distribution is our posterior, the
inverse-chi-squared distribution (or inverted-chi-square
distribution{[}1{]}) is a continuous probability distribution of a
positive-valued random variable. It is closely related to the
chi-squared distribution. It arises in Bayesian inference, where it can
be used as the prior and posterior distribution for an unknown variance
of the normal distribution.

the inverse chi distribution is The inverse chi-square distribution with
degrees of freedom n and scale parameter \(s^2\) is closely related to
the inverse gamma distribution with shape parameter \(\alpha\) = n/2 and
scale parameter \(\beta\) = 1/(2\(s^2\)). In fact, if X is
Inv-\(\chi^2\)(n,\(s^2\)) distributed, then Y = (n\(s^2\))/X is
Inv-\(\gamma\)(\(\alpha\),\(\beta\)) distributed, and vice versa. then
we use the function rinvgamma from r and we change on the pramters shape
and scale.

\begin{verbatim}
## Warning: Removed 118 rows containing non-finite values (stat_bin).
\end{verbatim}

\begin{verbatim}
## Warning: Removed 118 rows containing non-finite values (stat_density).
\end{verbatim}

\begin{verbatim}
## Warning: Removed 2 rows containing missing values (geom_bar).
\end{verbatim}

\includegraphics{Bayesian-Learning-Lab-1_files/figure-latex/lognormal-1.pdf}

\hypertarget{b-1}{%
\section{B}\label{b-1}}

We define The Gini coefficient as a measure of inequality in a
distribution, typically used to measure income inequality. It ranges
from 0 (perfect equality, where everyone has the same income) to 1
(perfect inequality, where one person has all the income). A Gini
coefficient of 0.5, for example, indicates that 50\% of the population
has 50\% of the total income, while the other 50\% of the population has
the remaining 50\% of the income.

\emph{Steps}:

1- We find the value of \(\phi(\sigma/\sqrt2)\) by using the values of
\(\sigma^2\) from the estimated values in A using the formula
\(\sigma/\sqrt2\). 2- We use the function \(pnorm\) to find the values
of \(\phi\) where: - q: the quantile(s) at which to evaluate the CDF.
mean: the mean of the normal distribution (default value is 0). - sd:
the standard deviation of the normal distribution (default value is 1).
- lower.tail: a logical value indicating whether to compute the lower
tail probability (TRUE, default) or the upper tail probability (FALSE).
- log.p: a logical value indicating whether to return the natural
logarithm of the probability density (TRUE) or the probability density
(FALSE, default). \emph{Note that}: Quantiles are points in a
probability distribution that divide the distribution into intervals of
equal probability.In our case we use the probabilities from the function
\(x=rinvgamma(nDraws, shape=n/2, scale=n*tau2/2)\)

3- We calculate the value of G by pluggin 1 and 2

\includegraphics{Bayesian-Learning-Lab-1_files/figure-latex/Gini-1.pdf}
\# C To find the 95\% CI we use the function \(qnorm()\)in R with mean
and Sd drived from the Gini distribution we found in the previous data.
here we indicates that 0,025 is the 2,5\% cut off point for both uppbe
and lower limits.

\begin{verbatim}
## 95% confidence interval: [ 0.16 , 0.51 ]
\end{verbatim}

\hypertarget{d}{%
\section{D}\label{d}}

We define the Highest Posterior Density (HPD) interval is a type of
confidence interval in Bayesian inference that contains the most
credible values for a parameter based on the observed data.
Specifically, it is the narrowest interval that contains a specified
proportion (usually 95\% or 99\%) of the posterior distribution of the
parameter. To do so First we generate a sample using the posterior mean
and Sd then We use the function HDI to find the 95\% CI from
library(HDInterval)

\begin{verbatim}
## Warning: package 'HDInterval' was built under R version 4.2.3
\end{verbatim}

\begin{verbatim}
## 95% equal tail interavl for G : [ 0.16 , 0.51 ]
\end{verbatim}

\begin{verbatim}
## 95% Highest Posterioe Density Interval for G: [ 0.16 , 0.5 ]
\end{verbatim}

\end{document}
