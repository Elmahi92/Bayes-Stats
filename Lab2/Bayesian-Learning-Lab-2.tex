% Options for packages loaded elsewhere
\PassOptionsToPackage{unicode}{hyperref}
\PassOptionsToPackage{hyphens}{url}
%
\documentclass[
]{article}
\usepackage{amsmath,amssymb}
\usepackage{lmodern}
\usepackage{iftex}
\ifPDFTeX
  \usepackage[T1]{fontenc}
  \usepackage[utf8]{inputenc}
  \usepackage{textcomp} % provide euro and other symbols
\else % if luatex or xetex
  \usepackage{unicode-math}
  \defaultfontfeatures{Scale=MatchLowercase}
  \defaultfontfeatures[\rmfamily]{Ligatures=TeX,Scale=1}
\fi
% Use upquote if available, for straight quotes in verbatim environments
\IfFileExists{upquote.sty}{\usepackage{upquote}}{}
\IfFileExists{microtype.sty}{% use microtype if available
  \usepackage[]{microtype}
  \UseMicrotypeSet[protrusion]{basicmath} % disable protrusion for tt fonts
}{}
\makeatletter
\@ifundefined{KOMAClassName}{% if non-KOMA class
  \IfFileExists{parskip.sty}{%
    \usepackage{parskip}
  }{% else
    \setlength{\parindent}{0pt}
    \setlength{\parskip}{6pt plus 2pt minus 1pt}}
}{% if KOMA class
  \KOMAoptions{parskip=half}}
\makeatother
\usepackage{xcolor}
\usepackage[margin=1in]{geometry}
\usepackage{color}
\usepackage{fancyvrb}
\newcommand{\VerbBar}{|}
\newcommand{\VERB}{\Verb[commandchars=\\\{\}]}
\DefineVerbatimEnvironment{Highlighting}{Verbatim}{commandchars=\\\{\}}
% Add ',fontsize=\small' for more characters per line
\usepackage{framed}
\definecolor{shadecolor}{RGB}{248,248,248}
\newenvironment{Shaded}{\begin{snugshade}}{\end{snugshade}}
\newcommand{\AlertTok}[1]{\textcolor[rgb]{0.94,0.16,0.16}{#1}}
\newcommand{\AnnotationTok}[1]{\textcolor[rgb]{0.56,0.35,0.01}{\textbf{\textit{#1}}}}
\newcommand{\AttributeTok}[1]{\textcolor[rgb]{0.77,0.63,0.00}{#1}}
\newcommand{\BaseNTok}[1]{\textcolor[rgb]{0.00,0.00,0.81}{#1}}
\newcommand{\BuiltInTok}[1]{#1}
\newcommand{\CharTok}[1]{\textcolor[rgb]{0.31,0.60,0.02}{#1}}
\newcommand{\CommentTok}[1]{\textcolor[rgb]{0.56,0.35,0.01}{\textit{#1}}}
\newcommand{\CommentVarTok}[1]{\textcolor[rgb]{0.56,0.35,0.01}{\textbf{\textit{#1}}}}
\newcommand{\ConstantTok}[1]{\textcolor[rgb]{0.00,0.00,0.00}{#1}}
\newcommand{\ControlFlowTok}[1]{\textcolor[rgb]{0.13,0.29,0.53}{\textbf{#1}}}
\newcommand{\DataTypeTok}[1]{\textcolor[rgb]{0.13,0.29,0.53}{#1}}
\newcommand{\DecValTok}[1]{\textcolor[rgb]{0.00,0.00,0.81}{#1}}
\newcommand{\DocumentationTok}[1]{\textcolor[rgb]{0.56,0.35,0.01}{\textbf{\textit{#1}}}}
\newcommand{\ErrorTok}[1]{\textcolor[rgb]{0.64,0.00,0.00}{\textbf{#1}}}
\newcommand{\ExtensionTok}[1]{#1}
\newcommand{\FloatTok}[1]{\textcolor[rgb]{0.00,0.00,0.81}{#1}}
\newcommand{\FunctionTok}[1]{\textcolor[rgb]{0.00,0.00,0.00}{#1}}
\newcommand{\ImportTok}[1]{#1}
\newcommand{\InformationTok}[1]{\textcolor[rgb]{0.56,0.35,0.01}{\textbf{\textit{#1}}}}
\newcommand{\KeywordTok}[1]{\textcolor[rgb]{0.13,0.29,0.53}{\textbf{#1}}}
\newcommand{\NormalTok}[1]{#1}
\newcommand{\OperatorTok}[1]{\textcolor[rgb]{0.81,0.36,0.00}{\textbf{#1}}}
\newcommand{\OtherTok}[1]{\textcolor[rgb]{0.56,0.35,0.01}{#1}}
\newcommand{\PreprocessorTok}[1]{\textcolor[rgb]{0.56,0.35,0.01}{\textit{#1}}}
\newcommand{\RegionMarkerTok}[1]{#1}
\newcommand{\SpecialCharTok}[1]{\textcolor[rgb]{0.00,0.00,0.00}{#1}}
\newcommand{\SpecialStringTok}[1]{\textcolor[rgb]{0.31,0.60,0.02}{#1}}
\newcommand{\StringTok}[1]{\textcolor[rgb]{0.31,0.60,0.02}{#1}}
\newcommand{\VariableTok}[1]{\textcolor[rgb]{0.00,0.00,0.00}{#1}}
\newcommand{\VerbatimStringTok}[1]{\textcolor[rgb]{0.31,0.60,0.02}{#1}}
\newcommand{\WarningTok}[1]{\textcolor[rgb]{0.56,0.35,0.01}{\textbf{\textit{#1}}}}
\usepackage{graphicx}
\makeatletter
\def\maxwidth{\ifdim\Gin@nat@width>\linewidth\linewidth\else\Gin@nat@width\fi}
\def\maxheight{\ifdim\Gin@nat@height>\textheight\textheight\else\Gin@nat@height\fi}
\makeatother
% Scale images if necessary, so that they will not overflow the page
% margins by default, and it is still possible to overwrite the defaults
% using explicit options in \includegraphics[width, height, ...]{}
\setkeys{Gin}{width=\maxwidth,height=\maxheight,keepaspectratio}
% Set default figure placement to htbp
\makeatletter
\def\fps@figure{htbp}
\makeatother
\setlength{\emergencystretch}{3em} % prevent overfull lines
\providecommand{\tightlist}{%
  \setlength{\itemsep}{0pt}\setlength{\parskip}{0pt}}
\setcounter{secnumdepth}{-\maxdimen} % remove section numbering
\ifLuaTeX
  \usepackage{selnolig}  % disable illegal ligatures
\fi
\IfFileExists{bookmark.sty}{\usepackage{bookmark}}{\usepackage{hyperref}}
\IfFileExists{xurl.sty}{\usepackage{xurl}}{} % add URL line breaks if available
\urlstyle{same} % disable monospaced font for URLs
\hypersetup{
  pdftitle={Bayesian learning Lab 2},
  pdfauthor={Mohamed Ali - Mohal954},
  hidelinks,
  pdfcreator={LaTeX via pandoc}}

\title{Bayesian learning Lab 2}
\author{Mohamed Ali - Mohal954}
\date{2023-05-10}

\begin{document}
\maketitle

\hypertarget{linear-and-polynomial-regression}{%
\subsection{Linear and polynomial
regression}\label{linear-and-polynomial-regression}}

The dataset TempLambohov.txt contains daily average temperatures (in
degree Celcius) at Lambohov, Linköping over the course of the year 2019.
The response variable is temp and the covariate is

\[
\begin{aligned}
time &= \frac{the \ number\ of\ days\ since\ the\ beginning\ of\ the\ year}{365}
\end{aligned}
\] A Bayesian analysis of the following quadratic regression model is to
be performed: \[
\begin{aligned}
temp &= \beta_0 + \beta_1.time \ + e,\ e\sim N(0,\sigma^2)
\end{aligned}
\]

To answer this question a conjugate prior for the linear regression
model will be used in which: The joint prior for
\(\beta \ and\  \sigma^2\): \[
\begin{aligned}
\beta|\sigma^2 &\sim \ \mathcal{N}(\mu_0,\sigma^2\Omega{_0}^{-1})\\ \sigma^2 &\sim Inv-\chi^2(v_0,\sigma^2)
\end{aligned}
\] While the posterior: \[
\begin{aligned}
\beta|\sigma^2,y &\sim \ \mathcal{N}(\mu_n,\sigma^2\Omega{_n}^{-1})\\ 
\sigma^2 &\sim Inv-\chi^2(v_n,\sigma{^2}{_n})\\
where:\\
\mu_n&= (X'X\Omega_0)^{-1}\ (X'X\hat\beta+\Omega_0\mu_0)\\
\Omega_n&=X'X+\Omega_0\\
v_n&= v_0+n\\
v_n\sigma^2&=v_n\sigma^2+(y'y+\mu_0'\Omega_0\mu_0-\mu_n'\Omega_n\mu_n)
\end{aligned}
\]

First we start by reading the files and then we Create the
covariate\_time vaiable as (the number of days sinvce the beginning of
the year / 365)

\begin{Shaded}
\begin{Highlighting}[]
\CommentTok{\#reading the files }
\NormalTok{df}\OtherTok{\textless{}{-}}\FunctionTok{read\_xlsx}\NormalTok{(}\StringTok{"Linkoping2022.xlsx"}\NormalTok{)}
\CommentTok{\#Creating the covariate\_time vaiable as }
\CommentTok{\#(the number of days sinvce the beginning of the year / 365)  }
\NormalTok{a}\OtherTok{\textless{}{-}}\NormalTok{ df}\SpecialCharTok{$}\NormalTok{datetime}
\CommentTok{\#begunning of the year}
\NormalTok{b}\OtherTok{\textless{}{-}} \StringTok{\textquotesingle{}2022{-}01{-}01\textquotesingle{}}
\NormalTok{a}\OtherTok{\textless{}{-}}\FunctionTok{format.POSIXlt}\NormalTok{(}\FunctionTok{strptime}\NormalTok{(a,}\StringTok{\textquotesingle{}\%Y{-}\%m{-}\%d\textquotesingle{}}\NormalTok{))}
\NormalTok{b}\OtherTok{\textless{}{-}}\FunctionTok{format.POSIXlt}\NormalTok{(}\FunctionTok{strptime}\NormalTok{(b,}\StringTok{\textquotesingle{}\%Y{-}\%m{-}\%d\textquotesingle{}}\NormalTok{))}
\CommentTok{\#time diff from the }
\NormalTok{x}\OtherTok{\textless{}{-}}\FunctionTok{as.vector}\NormalTok{(}\FunctionTok{difftime}\NormalTok{(a,b,}\AttributeTok{units=}\StringTok{\textquotesingle{}days\textquotesingle{}}\NormalTok{))}
\NormalTok{df}\SpecialCharTok{$}\NormalTok{cov\_tm}\OtherTok{\textless{}{-}}\NormalTok{x}\SpecialCharTok{/}\DecValTok{365}
\end{Highlighting}
\end{Shaded}

\hypertarget{a}{%
\section{A}\label{a}}

Use the conjugate prior for the linear regression model. The prior
hyperparameters \(\mu_0, \Omega_0, v_0, \sigma^2_0\) shall be set to
sensible values. Start with
\(\mu_0=(-10,100,-100)^T, \Omega_0=0.02.I_3, v_0=3, \sigma^2_0=2\).
Check if this prior agrees with your prior opinions by simulating draws
from the joint prior of all parameters and for every draw compute the
regression curve. This gives a collection of regression curves; one for
each draw from the prior. Does the collection of curves look reasonable?
If not, change the prior hyperparameters until the collection of prior
regression curves agrees with your prior beliefs about the regression
curve. {[}Hint: R package mvtnorm can be used and your \(inv-\chi^2\)
simulator of random draws from Lab 1.{]}

We assume to use a conjugate prior from the linear regression in Lec 5 ,
we have been given the prior hyperparamteres as follow:

\begin{Shaded}
\begin{Highlighting}[]
\CommentTok{\#We assume to use a conjugate prior from }
\CommentTok{\#the linear regression in Lec 5 , }
\CommentTok{\#we have been given the prior hyperparamteres as follow:}
\NormalTok{mu\_0}\OtherTok{=} \FunctionTok{as.matrix}\NormalTok{(}\FunctionTok{c}\NormalTok{(}\DecValTok{0}\NormalTok{,}\DecValTok{100}\NormalTok{,}\SpecialCharTok{{-}}\DecValTok{100}\NormalTok{),}\AttributeTok{ncol=}\DecValTok{3}\NormalTok{)}
\NormalTok{omega\_0}\OtherTok{=}\FloatTok{0.01}\SpecialCharTok{*}\FunctionTok{diag}\NormalTok{(}\DecValTok{3}\NormalTok{)}
\NormalTok{v\_0}\OtherTok{=}\DecValTok{1}
\NormalTok{segma2\_0}\OtherTok{=}\DecValTok{1}
\NormalTok{n}\OtherTok{=} \FunctionTok{length}\NormalTok{(df}\SpecialCharTok{$}\NormalTok{temp)}
\NormalTok{ndraws}\OtherTok{=}\DecValTok{10}
\end{Highlighting}
\end{Shaded}

We have the joint prior for beta and segma2 defined as
\(\beta|\sigma^2\sim\mathcal{N}(\mu_0,\sigma^2\Omega_{_0}^{-1})\) and
\(\sigma^2 \sim Inv-\chi^2(v_o,\sigma^2_0)\) follows
Inv-Chi(v0,sigma2\_0)

First we draw our random samaple from inv-chi2 using the below defined
function from Lec 3 slide 5

\begin{Shaded}
\begin{Highlighting}[]
\CommentTok{\# Step 1: Draw X \textasciitilde{} χ²(n {-} 1)}
\NormalTok{draw\_chi\_sq }\OtherTok{\textless{}{-}} \ControlFlowTok{function}\NormalTok{(n) \{}
  \FunctionTok{return}\NormalTok{(}\FunctionTok{rchisq}\NormalTok{(}\DecValTok{1}\NormalTok{, }\AttributeTok{df =}\NormalTok{ n }\SpecialCharTok{{-}} \DecValTok{1}\NormalTok{))}
\NormalTok{\}}
\CommentTok{\# Step 2: Compute σ² = (n {-} 1) * s² / X}
\NormalTok{compute\_sigma\_sq }\OtherTok{\textless{}{-}} \ControlFlowTok{function}\NormalTok{(n, segma2\_0, X) \{}
  \FunctionTok{return}\NormalTok{((n }\SpecialCharTok{{-}} \DecValTok{1}\NormalTok{) }\SpecialCharTok{*}\NormalTok{ segma2\_0 }\SpecialCharTok{/}\NormalTok{ X)}
\NormalTok{\}}

\CommentTok{\# simulation}
\NormalTok{segma\_estimation }\OtherTok{\textless{}{-}} \ControlFlowTok{function}\NormalTok{(n, mu\_0, segma2\_0, ndraws) \{}
\NormalTok{  results }\OtherTok{\textless{}{-}} \FunctionTok{c}\NormalTok{()}
  
  \ControlFlowTok{for}\NormalTok{ (i }\ControlFlowTok{in} \DecValTok{1}\SpecialCharTok{:}\NormalTok{ndraws) \{}
\NormalTok{    X }\OtherTok{\textless{}{-}} \FunctionTok{draw\_chi\_sq}\NormalTok{(n)}
\NormalTok{    sigma\_sq }\OtherTok{\textless{}{-}} \FunctionTok{compute\_sigma\_sq}\NormalTok{(n, segma2\_0, X)}
\NormalTok{    results[i] }\OtherTok{\textless{}{-}}\NormalTok{ sigma\_sq}
\NormalTok{  \}}
  \FunctionTok{return}\NormalTok{(results)}
\NormalTok{\}}

\NormalTok{sigma2}\OtherTok{\textless{}{-}}\FunctionTok{segma\_estimation}\NormalTok{(n, mu\_0, segma2\_0, ndraws)}
\end{Highlighting}
\end{Shaded}

Now we estimate the betas values using the formula
\(\beta|\sigma^2\sim\mathcal{N}(\mu_0,\sigma^2\Omega_{_0}^{-1})\) and we
fit the regression based ontemp = beta0 + beta1 time + beta 2 time\^{}2
+ erorr

*Note we have our error follows the normal distrbution by 0 and
\(\sigma^2\)

\begin{Shaded}
\begin{Highlighting}[]
\ControlFlowTok{for}\NormalTok{ (i }\ControlFlowTok{in} \DecValTok{1}\SpecialCharTok{:}\FunctionTok{length}\NormalTok{(sigma2)) \{}
\NormalTok{  e}\OtherTok{\textless{}{-}} \FunctionTok{rnorm}\NormalTok{(}\DecValTok{1}\NormalTok{,}\DecValTok{0}\NormalTok{,sigma2[i])}
\NormalTok{  res}\OtherTok{\textless{}{-}}\FunctionTok{rmvnorm}\NormalTok{(}\DecValTok{1}\NormalTok{,mu\_0,sigma2[i]}\SpecialCharTok{*}\NormalTok{omega\_0)}
\NormalTok{  temp}\OtherTok{=}\NormalTok{ x}\OtherTok{=}\NormalTok{res[}\DecValTok{1}\NormalTok{,}\DecValTok{1}\NormalTok{]}\SpecialCharTok{+}\NormalTok{res[}\DecValTok{1}\NormalTok{,}\DecValTok{2}\NormalTok{]}\SpecialCharTok{*}\NormalTok{df}\SpecialCharTok{$}\NormalTok{cov\_tm}\SpecialCharTok{+}\NormalTok{res[}\DecValTok{1}\NormalTok{,}\DecValTok{3}\NormalTok{]}\SpecialCharTok{*}\NormalTok{df}\SpecialCharTok{$}\NormalTok{cov\_tm}\SpecialCharTok{\^{}}\DecValTok{2}\SpecialCharTok{+}\NormalTok{e}
\NormalTok{  df[[}\FunctionTok{paste0}\NormalTok{(}\StringTok{"temp\_fit\_"}\NormalTok{,sigma2[i])]]}\OtherTok{\textless{}{-}}\NormalTok{temp }
\NormalTok{\}}
\end{Highlighting}
\end{Shaded}

After we done with the draws now for every draw compute the regression
curve

\begin{Shaded}
\begin{Highlighting}[]
\CommentTok{\# Define a vector of colors}
\NormalTok{colors }\OtherTok{\textless{}{-}}\FunctionTok{c}\NormalTok{(}\StringTok{"\#FCA311"}\NormalTok{, }\StringTok{"\#00FF00"}\NormalTok{, }\StringTok{"\#0000FF"}\NormalTok{, }\StringTok{"\#FFFF00"}\NormalTok{, }\StringTok{"\#00FFFF"}\NormalTok{,}
            \StringTok{"\#FF00FF"}\NormalTok{, }\StringTok{"\#800000"}\NormalTok{, }\StringTok{"\#008000"}\NormalTok{, }\StringTok{"\#000080"}\NormalTok{, }\StringTok{"\#808000"}\NormalTok{,}
            \StringTok{"\#800080"}\NormalTok{, }\StringTok{"\#008080"}\NormalTok{, }\StringTok{"\#808080"}\NormalTok{, }\StringTok{"\#FFC0CB"}\NormalTok{, }\StringTok{"\#FFA500"}\NormalTok{,}
            \StringTok{"\#FFD700"}\NormalTok{, }\StringTok{"\#A52A2A"}\NormalTok{, }\StringTok{"\#7FFF00"}\NormalTok{, }\StringTok{"\#FF1493"}\NormalTok{, }\StringTok{"\#00BFFF"}\NormalTok{)}

\CommentTok{\# Plot with different colored lines}
\NormalTok{plt }\OtherTok{\textless{}{-}} \FunctionTok{ggplot}\NormalTok{(df, }\FunctionTok{aes}\NormalTok{(}\AttributeTok{x =}\NormalTok{ cov\_tm, }\AttributeTok{y =}\NormalTok{ temp)) }\SpecialCharTok{+}
  \FunctionTok{geom\_point}\NormalTok{(}\FunctionTok{aes}\NormalTok{(}\AttributeTok{color =} \FunctionTok{factor}\NormalTok{(}\StringTok{\textquotesingle{}temp\textquotesingle{}}\NormalTok{)), }\AttributeTok{size =} \DecValTok{1}\NormalTok{)}

\ControlFlowTok{for}\NormalTok{ (i }\ControlFlowTok{in} \FunctionTok{names}\NormalTok{(df)[}\SpecialCharTok{{-}}\FunctionTok{c}\NormalTok{(}\DecValTok{1}\SpecialCharTok{:}\DecValTok{4}\NormalTok{, }\FunctionTok{ncol}\NormalTok{(df))]) \{}
\NormalTok{  plt }\OtherTok{\textless{}{-}}\NormalTok{ plt }\SpecialCharTok{+} \FunctionTok{geom\_line}\NormalTok{(}\FunctionTok{aes\_string}\NormalTok{(}\AttributeTok{y =}\NormalTok{ i, }\AttributeTok{color =} \FunctionTok{factor}\NormalTok{(i)), }\AttributeTok{linetype =} \DecValTok{1}\NormalTok{)}
\NormalTok{\}}

\CommentTok{\# Map colors to the lines}
\NormalTok{plt }\OtherTok{\textless{}{-}}\NormalTok{ plt }\SpecialCharTok{+}
  \FunctionTok{scale\_color\_manual}\NormalTok{(}\AttributeTok{values =}\NormalTok{ colors) }\SpecialCharTok{+}
  \FunctionTok{labs}\NormalTok{(}\AttributeTok{x =} \StringTok{\textquotesingle{}Time\textquotesingle{}}\NormalTok{, }\AttributeTok{y =} \StringTok{\textquotesingle{}Temp\textquotesingle{}}\NormalTok{,}\AttributeTok{color=}\StringTok{\textquotesingle{}Predictions with different Segma values\textquotesingle{}}\NormalTok{)}
\NormalTok{plt}
\end{Highlighting}
\end{Shaded}

\begin{center}\includegraphics{Bayesian-Learning-Lab-2_files/figure-latex/plot-1} \end{center}

Comparing the above chart with our prior beliefs which we can see by
running the polynomial model using the \emph{lm} function in R with df =
2 we can see the fitted regression line with to some extend follow our
regression line when \(\sigma^2\) is equal to \emph{1.037}.

\begin{Shaded}
\begin{Highlighting}[]
\NormalTok{degree }\OtherTok{\textless{}{-}} \DecValTok{2}  \CommentTok{\# Set the degree of the polynomial}
\NormalTok{x}\OtherTok{=}\NormalTok{ df}\SpecialCharTok{$}\NormalTok{cov\_tm}
\NormalTok{y}\OtherTok{=}\NormalTok{ df}\SpecialCharTok{$}\NormalTok{temp}
\NormalTok{model }\OtherTok{\textless{}{-}} \FunctionTok{lm}\NormalTok{(y }\SpecialCharTok{\textasciitilde{}} \FunctionTok{poly}\NormalTok{(x, degree, }\AttributeTok{raw =} \ConstantTok{TRUE}\NormalTok{))}
\NormalTok{df\_plt}\OtherTok{\textless{}{-}} \FunctionTok{data.frame}\NormalTok{(}\AttributeTok{x=}\NormalTok{x,}\AttributeTok{y=}\NormalTok{y)}
\NormalTok{z}\OtherTok{=}\FunctionTok{predict}\NormalTok{(model)}
\CommentTok{\# Print the model summary}


\CommentTok{\# Plot the data and regression line}
\NormalTok{plt }\OtherTok{\textless{}{-}} \FunctionTok{ggplot}\NormalTok{(df\_plt, }\FunctionTok{aes}\NormalTok{(}\AttributeTok{x =}\NormalTok{ x, }\AttributeTok{y =}\NormalTok{ y)) }\SpecialCharTok{+}
  \FunctionTok{geom\_point}\NormalTok{(}\AttributeTok{color =} \StringTok{"\#4E79A7"}\NormalTok{, }\AttributeTok{size =} \DecValTok{1}\NormalTok{)}\SpecialCharTok{+}
  \FunctionTok{geom\_line}\NormalTok{(}\FunctionTok{aes}\NormalTok{(}\AttributeTok{y =}\NormalTok{ z), }\AttributeTok{color =} \StringTok{"\#FCA311"}\NormalTok{,}\AttributeTok{size=}\DecValTok{1}\NormalTok{ ,}\AttributeTok{linetype =} \DecValTok{1}\NormalTok{)}\SpecialCharTok{+}
  \FunctionTok{labs}\NormalTok{(}\AttributeTok{x =} \StringTok{\textquotesingle{}Time\textquotesingle{}}\NormalTok{, }\AttributeTok{y =} \StringTok{\textquotesingle{}Temp\textquotesingle{}}
\NormalTok{       ,}\AttributeTok{title =}\StringTok{\textquotesingle{}Polynomial Regression with Dgree 2\textquotesingle{}}\NormalTok{)}
\NormalTok{plt}
\end{Highlighting}
\end{Shaded}

\begin{center}\includegraphics{Bayesian-Learning-Lab-2_files/figure-latex/ploy-1} \end{center}

\begin{Shaded}
\begin{Highlighting}[]
\FunctionTok{summary}\NormalTok{(model)}
\end{Highlighting}
\end{Shaded}

\begin{verbatim}
## 
## Call:
## lm(formula = y ~ poly(x, degree, raw = TRUE))
## 
## Residuals:
##      Min       1Q   Median       3Q      Max 
## -10.6557  -2.8525  -0.1874   2.5052  12.6580 
## 
## Coefficients:
##                              Estimate Std. Error t value Pr(>|t|)    
## (Intercept)                   -7.3039     0.6582  -11.10   <2e-16 ***
## poly(x, degree, raw = TRUE)1  83.1568     3.0492   27.27   <2e-16 ***
## poly(x, degree, raw = TRUE)2 -78.3093     2.9600  -26.46   <2e-16 ***
## ---
## Signif. codes:  0 '***' 0.001 '**' 0.01 '*' 0.05 '.' 0.1 ' ' 1
## 
## Residual standard error: 4.215 on 362 degrees of freedom
## Multiple R-squared:  0.6726, Adjusted R-squared:  0.6708 
## F-statistic: 371.9 on 2 and 362 DF,  p-value: < 2.2e-16
\end{verbatim}

Now we draw simulations using the modified \(\mu_0\) from the model
results we have

\begin{Shaded}
\begin{Highlighting}[]
\NormalTok{mu\_news}\OtherTok{=} \FunctionTok{as.matrix}\NormalTok{(}\FunctionTok{c}\NormalTok{(}\FloatTok{7.3}\NormalTok{,}\FloatTok{83.1}\NormalTok{,}\SpecialCharTok{{-}}\FloatTok{78.3}\NormalTok{),}\AttributeTok{ncol=}\DecValTok{3}\NormalTok{)}
\NormalTok{segma2\_new}\OtherTok{=}\DecValTok{10}
\NormalTok{sigma2}\OtherTok{\textless{}{-}}\FunctionTok{segma\_estimation}\NormalTok{(n,mu\_news, segma2\_new, ndraws)}

\ControlFlowTok{for}\NormalTok{ (i }\ControlFlowTok{in} \DecValTok{1}\SpecialCharTok{:}\FunctionTok{length}\NormalTok{(sigma2)) \{}
\NormalTok{  e}\OtherTok{\textless{}{-}} \FunctionTok{rnorm}\NormalTok{(}\DecValTok{1}\NormalTok{,}\DecValTok{0}\NormalTok{,sigma2[i])}
\NormalTok{  res}\OtherTok{\textless{}{-}}\FunctionTok{rmvnorm}\NormalTok{(}\DecValTok{1}\NormalTok{,mu\_news,sigma2[i]}\SpecialCharTok{*}\NormalTok{omega\_0)}
\NormalTok{  temp}\OtherTok{=}\NormalTok{ x}\OtherTok{=}\NormalTok{res[}\DecValTok{1}\NormalTok{,}\DecValTok{1}\NormalTok{]}\SpecialCharTok{+}\NormalTok{res[}\DecValTok{1}\NormalTok{,}\DecValTok{2}\NormalTok{]}\SpecialCharTok{*}\NormalTok{df}\SpecialCharTok{$}\NormalTok{cov\_tm}\SpecialCharTok{+}\NormalTok{res[}\DecValTok{1}\NormalTok{,}\DecValTok{3}\NormalTok{]}\SpecialCharTok{*}\NormalTok{df}\SpecialCharTok{$}\NormalTok{cov\_tm}\SpecialCharTok{\^{}}\DecValTok{2}\SpecialCharTok{+}\NormalTok{e}
\NormalTok{  df[[}\FunctionTok{paste0}\NormalTok{(}\StringTok{"temp\_fit\_"}\NormalTok{,sigma2[i])]]}\OtherTok{\textless{}{-}}\NormalTok{temp }
\NormalTok{\}}

\CommentTok{\# Define a vector of colors}
\NormalTok{colors }\OtherTok{\textless{}{-}}\FunctionTok{c}\NormalTok{(}\StringTok{"\#FCA311"}\NormalTok{, }\StringTok{"\#00FF00"}\NormalTok{, }\StringTok{"\#0000FF"}\NormalTok{, }\StringTok{"\#FFFF00"}\NormalTok{, }\StringTok{"\#00FFFF"}\NormalTok{,}
         \StringTok{"\#FF00FF"}\NormalTok{, }\StringTok{"\#800000"}\NormalTok{, }\StringTok{"\#008000"}\NormalTok{, }\StringTok{"\#000080"}\NormalTok{, }\StringTok{"\#808000"}\NormalTok{,}
        \StringTok{"\#800080"}\NormalTok{, }\StringTok{"\#008080"}\NormalTok{, }\StringTok{"\#808080"}\NormalTok{, }\StringTok{"\#FFC0CB"}\NormalTok{, }\StringTok{"\#FFA500"}\NormalTok{,}
       \StringTok{"\#FFD700"}\NormalTok{, }\StringTok{"\#A52A2A"}\NormalTok{, }\StringTok{"\#7FFF00"}\NormalTok{, }\StringTok{"\#FF1493"}\NormalTok{, }\StringTok{"\#00BFFF"}\NormalTok{)}

\CommentTok{\# Plot with different colored lines}
\NormalTok{plt }\OtherTok{\textless{}{-}} \FunctionTok{ggplot}\NormalTok{(df, }\FunctionTok{aes}\NormalTok{(}\AttributeTok{x =}\NormalTok{ cov\_tm, }\AttributeTok{y =}\NormalTok{ temp)) }\SpecialCharTok{+}
  \FunctionTok{geom\_point}\NormalTok{(}\FunctionTok{aes}\NormalTok{(}\AttributeTok{color =} \FunctionTok{factor}\NormalTok{(}\StringTok{\textquotesingle{}temp\textquotesingle{}}\NormalTok{)), }\AttributeTok{size =} \DecValTok{1}\NormalTok{)}

\ControlFlowTok{for}\NormalTok{ (i }\ControlFlowTok{in} \FunctionTok{names}\NormalTok{(df)[}\SpecialCharTok{{-}}\FunctionTok{c}\NormalTok{(}\DecValTok{1}\SpecialCharTok{:}\DecValTok{15}\NormalTok{, }\FunctionTok{ncol}\NormalTok{(df))]) \{}
\NormalTok{  plt }\OtherTok{\textless{}{-}}\NormalTok{ plt }\SpecialCharTok{+} \FunctionTok{geom\_line}\NormalTok{(}\FunctionTok{aes\_string}\NormalTok{(}\AttributeTok{y =}\NormalTok{ i, }\AttributeTok{color =} \FunctionTok{factor}\NormalTok{(i)), }\AttributeTok{linetype =} \DecValTok{1}\NormalTok{)}
\NormalTok{\}}

\CommentTok{\# Map colors to the lines}
\NormalTok{plt }\OtherTok{\textless{}{-}}\NormalTok{ plt }\SpecialCharTok{+}
  \FunctionTok{scale\_color\_manual}\NormalTok{(}\AttributeTok{values =}\NormalTok{ colors) }\SpecialCharTok{+}
  \FunctionTok{labs}\NormalTok{(}\AttributeTok{x =} \StringTok{\textquotesingle{}Time\textquotesingle{}}\NormalTok{, }\AttributeTok{y =} \StringTok{\textquotesingle{}Temp\textquotesingle{}}\NormalTok{,}\AttributeTok{color=}\StringTok{\textquotesingle{}Predictions with different Segma values\textquotesingle{}}\NormalTok{) }
\NormalTok{plt}
\end{Highlighting}
\end{Shaded}

\begin{center}\includegraphics{Bayesian-Learning-Lab-2_files/figure-latex/modefid mu-1} \end{center}

\hypertarget{b}{%
\section{B}\label{b}}

Write a function that simulate draws from the joint posterior
distribution of \(\beta_0, \beta_1, \beta_2,\sigma^2\). i- Plot a
histogram for each marginal posterior of the parameters. ii- Make a
scatter plot of the temperature data and overlay a curve for the
posterior median of the regression function
\(f(time)=E[temp|time]=\beta_0+\beta_1.time+\beta_2.time^2\), i.e.~the
median of \(f(time)\) is computed for every value of time. In addition,
overlay curves for the 95\% equal tail posterior probability intervals
of \(f(time)\), i.e.~the 2.5 and 97.5 posterior percentiles of
\(f(time)\) is computed for every value of time. Does the posterior
probability intervals contain most of the data points? Should they?

As we want to estimate the uncertainty in the model parameters.
Simulating from the joint posterior allows us to obtain a set of
plausible values for all the parameters in the model, taking into
account the observed data and prior information. to do so: we have our
non-informative prior: \[
\begin{aligned}
p(\beta,\sigma^2) \propto \sigma^-1
\end{aligned}
\] Our joint posterior of \(\beta\) and \(\sigma^2\): \[
\begin{aligned}
\beta|\sigma^2,y &\sim \mathcal{N}(\hat\beta,\sigma^2(X´X)^{-1})\\
\sigma^2|y &\sim Inv-\mathcal{\chi^2}(n-k,s^2)\\
where&:\\
k&=number\ of \ \beta s\ in \ our \ case \ 3\\
\hat\beta&= (X´X)^{-1} X´y\\
s^2&= \frac{1}{n-k}(y-X\hat\beta)´(y-X\hat\beta)
\end{aligned}
\] Thus to simulate from the joint posterior we need to simulate from:
1- \(p(\sigma|y\) 2- \(p(\beta|\sigma^2,y)\)

And then we find the marginal posterior of \(\beta\):
\[\beta|y\sim t_n-k(\hat\beta,s^2(X´X)^-1)\] There for to draw a sample
from the joint posterior distribution of
\(\beta_0,\beta_2,\beta_3\ and \ \sigma^2\): we need first to calculate
the value of \(\hat\beta= (X´X)^-1X´y\)

\begin{Shaded}
\begin{Highlighting}[]
\NormalTok{y}\OtherTok{\textless{}{-}}\FunctionTok{as.matrix}\NormalTok{(df}\SpecialCharTok{$}\NormalTok{temp)}
\NormalTok{x}\OtherTok{\textless{}{-}}\FunctionTok{as.matrix}\NormalTok{(}\FunctionTok{cbind}\NormalTok{(}\DecValTok{1}\NormalTok{,df}\SpecialCharTok{$}\NormalTok{cov\_tm,df}\SpecialCharTok{$}\NormalTok{cov\_tm}\SpecialCharTok{\^{}}\DecValTok{2}\NormalTok{))}
\NormalTok{n}\OtherTok{\textless{}{-}} \FunctionTok{length}\NormalTok{(y)}
\NormalTok{k}\OtherTok{=}\DecValTok{3}
\NormalTok{beta\_ht}\OtherTok{\textless{}{-}}\NormalTok{(}\FunctionTok{solve}\NormalTok{(}\FunctionTok{t}\NormalTok{(x)}\SpecialCharTok{\%*\%}\NormalTok{x))}\SpecialCharTok{\%*\%}\NormalTok{(}\FunctionTok{t}\NormalTok{(x)}\SpecialCharTok{\%*\%}\NormalTok{y)}
\NormalTok{s2}\OtherTok{\textless{}{-}}\NormalTok{(}\DecValTok{1}\SpecialCharTok{/}\NormalTok{(n}\SpecialCharTok{{-}}\NormalTok{k))}\SpecialCharTok{*}\FunctionTok{t}\NormalTok{((y}\SpecialCharTok{{-}}\NormalTok{(x}\SpecialCharTok{\%*\%}\NormalTok{beta\_ht)))}\SpecialCharTok{\%*\%}\NormalTok{(y}\SpecialCharTok{{-}}\NormalTok{(x}\SpecialCharTok{\%*\%}\NormalTok{beta\_ht))}
\end{Highlighting}
\end{Shaded}

Then we calculate the values of
\(\Omega_n \ ,v_n \ ,\mu_n \ and \ \sigma{^2}{_n}\) we use the same
format as in eq 1

\begin{Shaded}
\begin{Highlighting}[]
\NormalTok{omega\_n }\OtherTok{\textless{}{-}} \FunctionTok{t}\NormalTok{(x) }\SpecialCharTok{\%*\%}\NormalTok{ x}\SpecialCharTok{+}\NormalTok{omega\_0}
\NormalTok{df\_}\OtherTok{\textless{}{-}}\NormalTok{(n}\SpecialCharTok{{-}}\NormalTok{k)}
\NormalTok{lmbda\_}\OtherTok{\textless{}{-}}\NormalTok{s2}
\NormalTok{v\_n }\OtherTok{\textless{}{-}}\NormalTok{ v\_0 }\SpecialCharTok{+}\NormalTok{ n}
\NormalTok{mu\_n }\OtherTok{\textless{}{-}} \FunctionTok{solve}\NormalTok{(}\FunctionTok{t}\NormalTok{(x) }\SpecialCharTok{\%*\%}\NormalTok{ x }\SpecialCharTok{+}\NormalTok{ omega\_0) }\SpecialCharTok{\%*\%}\NormalTok{ (}\FunctionTok{t}\NormalTok{(x) }\SpecialCharTok{\%*\%}\NormalTok{ x }\SpecialCharTok{\%*\%}\NormalTok{ beta\_ht }\SpecialCharTok{+}\NormalTok{ omega\_0 }\SpecialCharTok{\%*\%}\NormalTok{ mu\_0)}
\NormalTok{sigma2\_n }\OtherTok{\textless{}{-}}\NormalTok{ (v\_0 }\SpecialCharTok{*}\NormalTok{ segma2\_0 }\SpecialCharTok{+}\NormalTok{ (}\FunctionTok{t}\NormalTok{(y) }\SpecialCharTok{\%*\%}\NormalTok{ y }\SpecialCharTok{+} \FunctionTok{t}\NormalTok{(mu\_0) }\SpecialCharTok{\%*\%}\NormalTok{ omega\_0}
                               \SpecialCharTok{\%*\%}\NormalTok{ mu\_0 }\SpecialCharTok{{-}} \FunctionTok{t}\NormalTok{(mu\_n) }\SpecialCharTok{\%*\%}\NormalTok{ omega\_n }\SpecialCharTok{\%*\%}\NormalTok{ mu\_n)) }\SpecialCharTok{/}\NormalTok{ v\_n}
\end{Highlighting}
\end{Shaded}

Now we generate the valuse of \(\beta\) from
\(t_{n-k}(\hat\beta,\ s^2(X´X)^{-1} )\) using \(\mu_n\) as our delta and
\(\sigma^2_n\) as sigma. in R we use function \emph{mvtnorm::rmvt}

\begin{Shaded}
\begin{Highlighting}[]
\NormalTok{res}\OtherTok{\textless{}{-}}\FunctionTok{data.frame}\NormalTok{(mvtnorm}\SpecialCharTok{::}\FunctionTok{rmvt}\NormalTok{(}\DecValTok{10000}\NormalTok{,}\AttributeTok{delta=}\NormalTok{mu\_n,}\AttributeTok{df=}\NormalTok{df\_,}\AttributeTok{sigma=}\NormalTok{sigma2\_n[}\DecValTok{1}\NormalTok{,}\DecValTok{1}\NormalTok{]}\SpecialCharTok{*}\FunctionTok{solve}\NormalTok{(}\FunctionTok{t}\NormalTok{(x)}\SpecialCharTok{\%*\%}\NormalTok{x)))}
\end{Highlighting}
\end{Shaded}

i/ Now we Plot a histogram for each marginal posterior of the parameters
\(\beta's\)

\begin{Shaded}
\begin{Highlighting}[]
\CommentTok{\#We store the value of betas in a df and }
\CommentTok{\#then we use this data to plot the histogarm of the betas}
\NormalTok{res}\OtherTok{\textless{}{-}}\FunctionTok{data.frame}\NormalTok{(mvtnorm}\SpecialCharTok{::}\FunctionTok{rmvt}\NormalTok{(}\DecValTok{10000}\NormalTok{,}\AttributeTok{delta=}\NormalTok{mu\_n,}
                              \AttributeTok{df=}\NormalTok{df\_,}\AttributeTok{sigma=}\NormalTok{sigma2\_n[}\DecValTok{1}\NormalTok{,}\DecValTok{1}\NormalTok{]}\SpecialCharTok{*}\FunctionTok{solve}\NormalTok{(}\FunctionTok{t}\NormalTok{(x)}\SpecialCharTok{\%*\%}\NormalTok{x)))}


\FunctionTok{names}\NormalTok{(res)}\OtherTok{\textless{}{-}}\FunctionTok{c}\NormalTok{(}\StringTok{\textquotesingle{}b\_0\textquotesingle{}}\NormalTok{,}\StringTok{\textquotesingle{}b\_1\textquotesingle{}}\NormalTok{,}\StringTok{\textquotesingle{}b\_2\textquotesingle{}}\NormalTok{)}
\NormalTok{plt1 }\OtherTok{\textless{}{-}} \FunctionTok{ggplot}\NormalTok{(res,}\FunctionTok{aes}\NormalTok{(}\AttributeTok{x =}\NormalTok{ b\_0)) }\SpecialCharTok{+}\FunctionTok{geom\_histogram}\NormalTok{(}\FunctionTok{aes}\NormalTok{(}\AttributeTok{y=}\NormalTok{..density..),}
                  \AttributeTok{linetype=}\DecValTok{1}\NormalTok{,}
                  \AttributeTok{fill=}\StringTok{\textquotesingle{}\#14213D\textquotesingle{}}\NormalTok{,}\AttributeTok{binwidth =} \FloatTok{0.2}\NormalTok{)}\SpecialCharTok{+}
  \FunctionTok{labs}\NormalTok{(}\AttributeTok{x=}\StringTok{\textquotesingle{}Beta 0\textquotesingle{}}\NormalTok{,}\AttributeTok{y=}\StringTok{\textquotesingle{} \textquotesingle{}}\NormalTok{,}\AttributeTok{title =}\StringTok{\textquotesingle{}Marginal posterior of the parameters\textquotesingle{}}\NormalTok{)}\SpecialCharTok{+}
                \FunctionTok{stat\_function}\NormalTok{(}\AttributeTok{fun =}\NormalTok{ dnorm, }\AttributeTok{args =} \FunctionTok{list}\NormalTok{(}\AttributeTok{mean =} \FunctionTok{mean}\NormalTok{(res}\SpecialCharTok{$}\NormalTok{b\_0),}
                            \AttributeTok{sd =} \FunctionTok{sd}\NormalTok{(res}\SpecialCharTok{$}\NormalTok{b\_0)),}
                \AttributeTok{color =} \StringTok{"\#FCA311"}\NormalTok{, }\AttributeTok{size =} \DecValTok{1}\NormalTok{)}

\NormalTok{plt2 }\OtherTok{\textless{}{-}} \FunctionTok{ggplot}\NormalTok{(res,}\FunctionTok{aes}\NormalTok{(}\AttributeTok{x =}\NormalTok{ b\_1)) }\SpecialCharTok{+}\FunctionTok{geom\_histogram}\NormalTok{(}\FunctionTok{aes}\NormalTok{(}\AttributeTok{y=}\NormalTok{..density..),}
                                                 \AttributeTok{linetype=}\DecValTok{1}\NormalTok{,}
                                                 \AttributeTok{fill=}\StringTok{\textquotesingle{}\#14213D\textquotesingle{}}\NormalTok{,}\AttributeTok{binwidth =} \FloatTok{0.4}\NormalTok{)}\SpecialCharTok{+}
  \FunctionTok{labs}\NormalTok{(}\AttributeTok{x=}\StringTok{\textquotesingle{}Beta 1\textquotesingle{}}\NormalTok{,}\AttributeTok{y=}\StringTok{\textquotesingle{} \textquotesingle{}}\NormalTok{)}\SpecialCharTok{+}
  \FunctionTok{stat\_function}\NormalTok{(}\AttributeTok{fun =}\NormalTok{ dnorm, }\AttributeTok{args =} \FunctionTok{list}\NormalTok{(}\AttributeTok{mean =} \FunctionTok{mean}\NormalTok{(res}\SpecialCharTok{$}\NormalTok{b\_1),}
                                         \AttributeTok{sd =} \FunctionTok{sd}\NormalTok{(res}\SpecialCharTok{$}\NormalTok{b\_1)),}
                \AttributeTok{color =} \StringTok{"\#FCA311"}\NormalTok{, }\AttributeTok{size =} \DecValTok{1}\NormalTok{)}
\NormalTok{plt3 }\OtherTok{\textless{}{-}} \FunctionTok{ggplot}\NormalTok{(res,}\FunctionTok{aes}\NormalTok{(}\AttributeTok{x =}\NormalTok{ b\_2)) }\SpecialCharTok{+}\FunctionTok{geom\_histogram}\NormalTok{(}\FunctionTok{aes}\NormalTok{(}\AttributeTok{y=}\NormalTok{..density..),}
                                                 \AttributeTok{linetype=}\DecValTok{1}\NormalTok{,}
                                                 \AttributeTok{fill=}\StringTok{\textquotesingle{}\#14213D\textquotesingle{}}\NormalTok{,}\AttributeTok{binwidth =} \FloatTok{0.5}\NormalTok{)}\SpecialCharTok{+}
  \FunctionTok{labs}\NormalTok{(}\AttributeTok{x=}\StringTok{\textquotesingle{}Beta 2\textquotesingle{}}\NormalTok{,}\AttributeTok{y=}\StringTok{\textquotesingle{} \textquotesingle{}}\NormalTok{,)}\SpecialCharTok{+}
  \FunctionTok{stat\_function}\NormalTok{(}\AttributeTok{fun =}\NormalTok{ dnorm, }\AttributeTok{args =} \FunctionTok{list}\NormalTok{(}\AttributeTok{mean =} \FunctionTok{mean}\NormalTok{(res}\SpecialCharTok{$}\NormalTok{b\_2),}
                                         \AttributeTok{sd =} \FunctionTok{sd}\NormalTok{(res}\SpecialCharTok{$}\NormalTok{b\_2)),}
                \AttributeTok{color =} \StringTok{"\#FCA311"}\NormalTok{, }\AttributeTok{size =} \DecValTok{1}\NormalTok{)}

\NormalTok{plt1}\SpecialCharTok{+}\NormalTok{plt2}\SpecialCharTok{+}\NormalTok{plt3}
\end{Highlighting}
\end{Shaded}

\includegraphics{Bayesian-Learning-Lab-2_files/figure-latex/marginal posterior-1.pdf}
ii/ Make a scatter plot of the temperature data and overlay a curve for
the posterior median of the regression function
\(f(time)\ = \ E[temp|time]\ =\ \beta_0+\beta_1.time+\beta_2.time^2\).
First we need to calculate the \(f(time)\) by using the results from i
we can define:

\begin{Shaded}
\begin{Highlighting}[]
\CommentTok{\# Calculating the median value point}
\NormalTok{median}\OtherTok{=}\FunctionTok{as.matrix}\NormalTok{(}\FunctionTok{apply}\NormalTok{(res, }\DecValTok{2}\NormalTok{, median))}

\CommentTok{\# we fint the regression  model P(time)=beta\_0+beta\_1*time+beta\_2*time2}
\NormalTok{predicted\_response }\OtherTok{\textless{}{-}}\NormalTok{ x}\SpecialCharTok{\%*\%}\NormalTok{ median}

\CommentTok{\#storing the median values}
\NormalTok{posterior\_median }\OtherTok{\textless{}{-}} \FunctionTok{apply}\NormalTok{(predicted\_response, }\DecValTok{1}\NormalTok{, median)}

\CommentTok{\#Finding the predicted interval}
\NormalTok{prd\_int }\OtherTok{\textless{}{-}} \FunctionTok{data.frame}\NormalTok{(}\AttributeTok{nrow =}\NormalTok{ n, }\AttributeTok{nrow =} \DecValTok{2}\NormalTok{)}
\FunctionTok{colnames}\NormalTok{(prd\_int) }\OtherTok{\textless{}{-}} \FunctionTok{c}\NormalTok{(}\StringTok{"CI\_lower"}\NormalTok{,}\StringTok{"CI\_upper"}\NormalTok{)}
\NormalTok{preds}\OtherTok{\textless{}{-}} \FunctionTok{as.matrix}\NormalTok{(res)}\SpecialCharTok{\%*\%}\FunctionTok{t}\NormalTok{(x)}
\ControlFlowTok{for}\NormalTok{(i }\ControlFlowTok{in} \DecValTok{1}\SpecialCharTok{:}\FunctionTok{nrow}\NormalTok{(x))\{}
\NormalTok{  data\_t }\OtherTok{\textless{}{-}}\NormalTok{ preds[,i]}
  \CommentTok{\#Here we have 95\% CI using the function quantile}
\NormalTok{  prd\_int[i,] }\OtherTok{\textless{}{-}} \FunctionTok{quantile}\NormalTok{(data\_t, }\AttributeTok{probs =} \FunctionTok{c}\NormalTok{(}\FloatTok{0.05}\NormalTok{,}\FloatTok{0.95}\NormalTok{))}
\NormalTok{\}}
\end{Highlighting}
\end{Shaded}

by using the results from above we can fit the curve line for the
posterior median and 95\% CI on the scatter plot of the temperature data
as below:

\begin{Shaded}
\begin{Highlighting}[]
\CommentTok{\# Storing the data in one data frame}

\NormalTok{plt\_df}\OtherTok{=}\FunctionTok{data.frame}\NormalTok{(}\AttributeTok{x=}\NormalTok{df}\SpecialCharTok{$}\NormalTok{cov\_tm,}\AttributeTok{y=}\NormalTok{df}\SpecialCharTok{$}\NormalTok{temp,}\AttributeTok{med=}\NormalTok{posterior\_median)}
\NormalTok{plt\_df}\OtherTok{=} \FunctionTok{cbind}\NormalTok{(plt\_df,prd\_int)}
\CommentTok{\# Calculate posterior median of the predicted response}


\NormalTok{plt }\OtherTok{\textless{}{-}} \FunctionTok{ggplot}\NormalTok{(plt\_df, }\FunctionTok{aes}\NormalTok{(}\AttributeTok{x =}\NormalTok{ x, }\AttributeTok{y =}\NormalTok{ y)) }\SpecialCharTok{+}
  \FunctionTok{geom\_point}\NormalTok{(}\AttributeTok{color =} \StringTok{"\#14213D"}\NormalTok{, }\AttributeTok{size =} \FloatTok{1.5}\NormalTok{)}\SpecialCharTok{+}
  \FunctionTok{geom\_line}\NormalTok{(}\FunctionTok{aes}\NormalTok{(}\AttributeTok{y =}\NormalTok{ med), }\AttributeTok{color =} \StringTok{"\#F28E2B"}\NormalTok{, }\AttributeTok{linetype =} \DecValTok{1}\NormalTok{,}\AttributeTok{size=}\FloatTok{1.5}\NormalTok{)}\SpecialCharTok{+}
  \FunctionTok{geom\_ribbon}\NormalTok{(}\FunctionTok{aes}\NormalTok{(}\AttributeTok{ymin =}\NormalTok{ CI\_lower, }\AttributeTok{ymax =}\NormalTok{ CI\_upper)}
\NormalTok{              , }\AttributeTok{alpha =} \FloatTok{0.5}\NormalTok{,}\AttributeTok{fill =} \StringTok{"\#EDC948"}\NormalTok{)}\SpecialCharTok{+}
  \FunctionTok{labs}\NormalTok{(}\AttributeTok{x =} \StringTok{\textquotesingle{}Time\textquotesingle{}}\NormalTok{, }\AttributeTok{y =} \StringTok{\textquotesingle{}Temp\textquotesingle{}}
\NormalTok{       ,}\AttributeTok{title =}\StringTok{\textquotesingle{}The posterior median Curve and 95\% CI\textquotesingle{}}\NormalTok{) }
\NormalTok{plt}
\end{Highlighting}
\end{Shaded}

\includegraphics{Bayesian-Learning-Lab-2_files/figure-latex/curve line for the posterior-1.pdf}

\hypertarget{c}{%
\section{C}\label{c}}

It is of interest to locate the time with the highest expected
temperature (i.e.~the time where \(f(time)\) is maximal). Let's call
this value \(\bar x\). Use the simulated draws in (b) to simulate from
the posterior distribution of \(\bar x\). {[}Hint: the regression curve
is a quadratic polynomial. Given each posterior draw of
\(\beta_0, \beta_1, \beta_2\), you can find a simple formula for
\(\bar x\).{]}

To simulate from the posterior distribution of the highest expected
temperature \(\bar X\) we can use the all possible prediction values
that we generated in the task before, then by taking the max value we
get the point point wise highest expected temperature over time:

\begin{Shaded}
\begin{Highlighting}[]
\CommentTok{\# Storing the data in one data frame}
\CommentTok{\#Initite the storing vector}
\NormalTok{het}\OtherTok{\textless{}{-}}\FunctionTok{c}\NormalTok{()}

\CommentTok{\#Startingh the for loop }
\ControlFlowTok{for}\NormalTok{ (i }\ControlFlowTok{in} \DecValTok{1}\SpecialCharTok{:}\FunctionTok{nrow}\NormalTok{(x)) \{}
\NormalTok{  het[i]}\OtherTok{\textless{}{-}}\FunctionTok{max}\NormalTok{(preds[,i]) }
\NormalTok{\}}

\CommentTok{\# binding the data into te ploting data frame}

\NormalTok{plt\_df}\OtherTok{=} \FunctionTok{cbind}\NormalTok{(plt\_df,het)}

\CommentTok{\#Ploting the data}

\NormalTok{plt }\OtherTok{\textless{}{-}} \FunctionTok{ggplot}\NormalTok{(plt\_df, }\FunctionTok{aes}\NormalTok{(}\AttributeTok{x =}\NormalTok{ x, }\AttributeTok{y =}\NormalTok{ y)) }\SpecialCharTok{+}
  \FunctionTok{geom\_point}\NormalTok{(}\AttributeTok{color =} \StringTok{"\#14213D"}\NormalTok{, }\AttributeTok{size =} \FloatTok{1.5}\NormalTok{)}\SpecialCharTok{+}
  \FunctionTok{geom\_line}\NormalTok{(}\FunctionTok{aes}\NormalTok{(}\AttributeTok{y =}\NormalTok{ het), }\AttributeTok{color =} \StringTok{"\#59A14F"}\NormalTok{, }\AttributeTok{linetype =} \DecValTok{1}\NormalTok{,}\AttributeTok{size=}\FloatTok{1.5}\NormalTok{)}\SpecialCharTok{+}
  \FunctionTok{geom\_line}\NormalTok{(}\FunctionTok{aes}\NormalTok{(}\AttributeTok{y =}\NormalTok{ med), }\AttributeTok{color =} \StringTok{"\#F28E2B"}\NormalTok{, }\AttributeTok{linetype =} \DecValTok{1}\NormalTok{,}\AttributeTok{size=}\FloatTok{1.5}\NormalTok{)}\SpecialCharTok{+}
  \FunctionTok{geom\_ribbon}\NormalTok{(}\FunctionTok{aes}\NormalTok{(}\AttributeTok{ymin =}\NormalTok{ CI\_lower, }\AttributeTok{ymax =}\NormalTok{ CI\_upper)}
\NormalTok{              , }\AttributeTok{alpha =} \FloatTok{0.5}\NormalTok{,}\AttributeTok{fill =} \StringTok{"\#EDC948"}\NormalTok{)}\SpecialCharTok{+}
  \FunctionTok{labs}\NormalTok{(}\AttributeTok{x =} \StringTok{\textquotesingle{}Time\textquotesingle{}}\NormalTok{, }\AttributeTok{y =} \StringTok{\textquotesingle{}Temp\textquotesingle{}}
\NormalTok{       ,}\AttributeTok{title =}\StringTok{\textquotesingle{}The posterior median Curve, }
\StringTok{       95\% CI and Highest Expected Temperature\textquotesingle{}}
\NormalTok{       ,}\AttributeTok{color =} \StringTok{"Line Legend"}\NormalTok{) }\SpecialCharTok{+}
  \FunctionTok{scale\_color\_manual}\NormalTok{(}\AttributeTok{values =} \FunctionTok{c}\NormalTok{(}\StringTok{"\#14213D"}\NormalTok{,}\StringTok{"\#59A14F"}\NormalTok{,}\StringTok{"\#F28E2B"}\NormalTok{,}\StringTok{"\#EDC948"}\NormalTok{)}
\NormalTok{                     , }\AttributeTok{labels =} \FunctionTok{c}\NormalTok{(}\StringTok{"1"}\NormalTok{,}\StringTok{"2"}\NormalTok{,}\StringTok{"3"}\NormalTok{,}\StringTok{"4"}\NormalTok{))}\SpecialCharTok{+}
  \FunctionTok{theme}\NormalTok{(}\AttributeTok{legend.position=}\StringTok{"bottom"}\NormalTok{)}
\NormalTok{plt}
\end{Highlighting}
\end{Shaded}

\includegraphics{Bayesian-Learning-Lab-2_files/figure-latex/highest expected temperature-1.pdf}

\hypertarget{d}{%
\section{D}\label{d}}

Say now that you want to estimate a polynomial regression of order 8,
but you suspect that higher order terms may not be needed, and you worry
about overfitting the data. Suggest a suitable prior that mitigates this
potential problem. You do not need to compute the posterior. Just write
down your prior. {[}Hint: the task is to specify \(\mu_0\) and
\(\Omega_0\) in a suitable way.{]}

To mitigate the problem of over fitting when estimating a polynomial
resgression of order 10 without being worry of the over fitting, one can
use Smoothness/Shrinkage/Regularization of the prior by introducing
\(\lambda\) a penalization parameter (See lec 5 Slide 9 Lasso) in this
method we have:

\[
\begin{aligned}
\beta_j|\sigma^2&\sim N(o,\frac{\sigma^2}{\lambda})
\end{aligned}
\] Here we have a large values of \(\lambda\) gives smoother fit. More
shrinkage. where: \[
\begin{aligned}
\mu_0&=0\\
\Omega_0&= \lambda I
\end{aligned}
\] Which equivalent to \emph{Penalized Likelihood}: \[
\begin{aligned}
-2log\ p(\beta|\sigma^2,y,x) &\propto(y-X\beta)'(y-X\beta)+\lambda\beta'\beta
\end{aligned}
\] Thus, the Posterior mean/mode gives ridge regressoin estimator: \[
\begin{aligned}
\tilde \beta &= (X´X+\lambda I)^{-1}X´y\\
&if \ X´X= I\\
&Then\\
\tilde \beta &= \frac{1}{(1+\lambda)}\hat\beta
\end{aligned}
\] We might also be interested to determine the lambda, which could be
by performing a cross validation on the test data pf using the Bayesian
inference, where to us a prior for \(\lambda\). we have this
hierarchical setup: \[
\begin{aligned}
y|\beta,\sigma^2,x &\sim N(X\beta,\sigma^2I_n)\\
\beta|\sigma^2,\lambda&\sim N(0,\sigma^2\lambda^{-1}I_m)\\
\sigma^2&\sim Inv-\chi^2(v_o,\sigma_o^2)\\
\lambda&\sim Inv-\chi^2(\eta_0,\lambda_0)\\
so,\ \mu_o&=0,\  \Omega=\lambda I_m
\end{aligned}
\] and we have the joint posterior of \(\beta\), \(\sigma^2\), and
\(\lambda\) is: \[
\begin{aligned}
\beta|\sigma^2, \lambda,y &\sim N(\mu_n,\sigma^2\Omega_n^{-1})\\
\sigma^2|\lambda,y &\sim Inv-\chi^2(v_n,\sigma^2_n)\\
p(\lambda|y) &\propto \sqrt\frac{|\Omega_0|}{|X^TX+\Omega_0|}(\frac{v_n\sigma_n^2}{2})^{\frac{-v_n}{2}}.p(\lambda)\\
where,\ \Omega_0&=\lambda I_m \ and\ p(\lambda) \ is\  the\  prior\  for\  \lambda \ and:\\
\mu_n&= (X^TX+\Omega_0)^{-1} X^Ty\\
\Omega_n&=X^TX+\Omega_0\\
v_n&=v_0+n\\
v_n\sigma_n^2&=v_0\sigma_0^2+y^Ty-\mu_0^T\Omega_n\mu_n
\end{aligned}
\]

\end{document}
